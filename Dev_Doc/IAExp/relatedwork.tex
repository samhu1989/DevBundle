

\section{Related Work}


Many techniques have been proposed to generate static 3D indoor scenes in computer graphics.
Though none of them focus on dynamic scene analysis like our system, they provide valuable reference on the underlying techniques.

\paragraph{Reconstruction from RGBD Images.}
%
For static scenes, KinectFusion~\cite{kinectfusion11} enables the real-time reconstruction by holding and moving a depth camera.
%
For large-scale indoor scenes with multiple rooms, reconstructing a dense 3D model from the noisy and incomplete scanned range data typically involves registration of point clouds in different views and a global optimization to reduce gaps in a large scene~\cite{MuseumCSG12,Hen12RGB}.
Their goal is mainly to generate high-quality point clouds but without semantic analysis of the objects appear in the scene.
Recently, object classification is employed to assist modeling for massive indoor scenes that containing many instances of chairs, desks, etc.
Koppula et al.~\shortcite{Koppula/etal/11} first introduce the learning algorithm to understand the RGBD data of an indoor scene.
To further reconstruct the 3{D} model for a cluttered indoor scene, 3{D} model databases can be used as template by searching for similar 3D model and then fitting the template to the scanned data~\cite{shao12,NanIndoor2012}.
\cite{kmyg_acquireIndoor_sigga12} do not manually collect 3{D} models to build the database.
The template model is reconstructed by scanning the same object in different configuration.
Each model has an additional presentation by geometric primitives.
\cite{shao12} trains the class model based on geometry and appearance features to segment and label the RGBD data captured under sparse views.
By learned an initial model for each class of object in indoor environments from a pre-labelled database, the model are refined progressively with user-refined segmentation results.
The 3{D} model can be generated by placing the most similar model in the database according to the RGBD data.
If objects move in a scene, they can be detected and reposed by segmented and classified based on the learned model from previously reconstructed model \cite{IndoorUpdatingPG14}.
Different with these techniques, we pay more attentions on analyzing the object behaviors from the dynamic range data.

%

\paragraph{Reconstruction from Sequential Point clouds.}

Many techniques have been proposed to reconstruct the object surfaces from the range data sequences.
\cite{Wand2007} uses a \emph{statistical framework} to reconstruct the geometry from real-time range scanning.
Each frame is divided into 3{D} pieces. A statistical model is used to iteratively merge adjacent frames by aligning pieces and optimizing their shapes.
%\emph{input: noisy and incomplete point clouds, large portion of missing data. output: animated meshes with dense correspondences.}
However, some geometric artifacts remain due to structured outliers and in some boundary regions.
\cite{chang11global} presents a global registration algorithm to reconstruct \emph{articulated 3D models} from dynamic range scan sequences.
The surface motion is modeled by a reduced deformable model.
Joints and skinning weights are solved in the system to register point clouds in different poses.
\xj{We may also consider the furniture objects in indoor environments as articulated models, whose shapes under different poses can be deformed through connectors like hinge, slide, and so on. }
%
A new formulation of the ICP algorithm is proposed using sparse inducing norms~\cite{sparseicp_sgp13}.
While it achieves superior registration result on the data with outliers and missing region, only rigid alignment is handled.
%
A proactive capturing is employed by asking the user to move the objects to capture both interior and exterior of a scene~\cite{YanSiggraph14}. The correspondence between adjacent frames is built first then segmentation. 
Xu et al.~\shortcite{xu_siga15} employ a robot to move objects during scene reconstruction so that the ambiguities in object structures can be solved from the dynamic data.
%
All these techniques take the advantage of the differences caused by motion to extract valuable and semantic information of object structure.
We do not only take the motion information in object modeling, but also take its advantage of implicitly encoding object behavior in a scene. 

\paragraph{Functionality Analysis using Context.}
Many techniques of functional analysis of one category of objects using 3D model collections have been proposed~\cite{Huang2014,Su2014_3dattributes}. 
Besides of functionality analysis of a single object, interaction between objects has drawn more and more attentions.
The Icon descriptor is proposed to represent the functionality of 3D objects with its context ~\cite{HuICON2015}.
%
Human action is involved to establish the correlations between the geometry and functionality of a region \cite{savva2014scenegrok}.  
%
With the association of object arrangements and human activities, novel 3D scenes can be synthesized towards specific functions~\cite{fisher2015actsynth}.  
%
Inspired by these methods, we put our effort on behavior recovery, but from more challenging data.
%
The point clouds can be easily to capture using RGBD cameras. However, the noisy and incompleteness brings tremendous challenges for extracting accurate object correspondences.






\paragraph{Data-Driven Furniture Layout.}

The general way producing the layout of furniture objects is to model a set of design rules and then to optimize an energy function given constraints by individuals.
%
\cite{Merrell11} formulates a group of layout guidelines in a density function according to professional manuals on furniture layout.
When the user specifies the room shape and an initial arrangement of the set of furniture to be placed in the room, this system generates a number of layout suggestions by a hardware-accelerated Monte Carlo sampler.
%
Instead of manually define the layout guidelines, the hierarchical and spatial relationships of the furniture objects can be learned from a set of examples~\cite{craigyu2011furniture}.
Assembling these relationships and other ergonomic factors into a cost function, multiple arrangements can be yielded quickly by simulated annealing using a Metropolis-Hastings state search step.
%
In these methods, manual labours are required in modeling the design rules and providing an initial layout.
%
Fisher et al.~\shortcite{Fisher2012} trains a probabilistic model for indoor scenes from a small number of examples.
A variety of indoor scenes can be automatically synthesized from a few of user specified examples.
%
Indoor scenes bring more difficulties for scene analysis because there are always many cluttered objects in different scales, shapes, and functions.
A focal-driven analysis and organization framework is presented for heterogeneous collections of indoor scenes~\cite{xu_sig14}.
They develop a co-analysis algorithm which interleaves frequent patten mining and subspace clustering.
The interrelations between objects play important role during furniture arrangement in these systems.
However, the 3D scene models takes many efforts to collect for training.
In comparison, our system provides an efficient framework to generate 3d model examples for many further applications.






\comment{
\paragraph{Co-analysis of shape, functions in a large database of 3D object models}

With the growth of 3D shape databases on the Internet, many techniques have been proposed for co-analysis in a large shape collection of the same object category.
A series of geometry processing tasks such as model segmentation, shape retrieval, and shape synthesis.
%
Point-to-point networks are used to represent the shape correspondence between shapes~\cite{Rustamov2013}.
To better explore the shape space, \cite{Huang2014} propose a framework for computing consistent functional maps within heterogeneous shape collections.
Cycle-consistency of the functional map network largely reduce the noise correspondences.
Based on the continuous nature of functional maps, the proposed framework outperforms point-based representation in shape interpolation, shape retrieval and classifications for both man-made and organic shapes.
%
Large 3d model collection can also help recovering the depth map for a single image~\cite{Su2014_3dattributes}.
With a non-rigid registration formulation, the image is popped-up to minimize the distance between corresponding points in the image and similar 3d shapes in the database.
%
However, all these techniques focus on the geometry characteristics within one object category.
In a cluttered environment, the inter-connection between different types of objects has not been investigated yet.

\paragraph{RGBD image understanding}

Though many techniques have been proposed to model a cluttered indoor environment based on databases, little work has been done for the physical interactions between objects. \cite{Silberman:CCV12} introduce an framework to segment the RGBD image and infer the support relationship between objects in a cluttered indoor scene.
A dataset is also provided for various tasks, such as recognition, segmentation and relationship inference.




\paragraph{Indoor Scenes, Arrangement, layout}

An indoor scene is represented by a graph of its constituent objects.
Two types of spatial relationships between different objects are modeled as the edges in the graph.

\xj{In comparison, our system supports more object interactions to describe the semantic relationship without pre-labeled the object categories.}
}
