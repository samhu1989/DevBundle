\documentclass{onrannual}

% provide additional characters
\usepackage{textcomp}

% Bibliography style; customize as appropriate, or remove if you don't use
% BibTeX or have no citations.
\usepackage[round]{natbib}
\setlength{\bibhang}{0pt}

% allow graphics inclusion
\usepackage{graphicx}

% for highlighting messages
\usepackage{color}

% optional: use hyperref to add in some PDF metadata
\hypersetup {
    pdftitle={Project Annual Report},
    pdfsubject={FY16 Annual Report},
    pdfauthor={Author},
    pdfkeywords={annual report, LaTeX, ONR}
}

%%
%% Two ways to specify an author.  Note the manual paragraph requirements if \affil is used.
%%

\author{SamHu}
\affil{%
%Academic Institute \\
%Laboratory 1 \\
%Somewhere, WA 11111 \\
%% NOTE: the trailing \\ is required at the end of this last line to separate authors
phone: +8615656928957  email: \href{sy891228@mail.ustc.edu.cn}{sy891228@mail.ustc.edu.cn} \\}

%% NOTE: distribution statements taken from 2009 guidance, and should be updated as appropriate for the current year.
%% One of these statements is required. so modify or uncomment as appropriate.
\distribution{DISTRIBUTION STATEMENT A: Distribution approved for public release; distribution is unlimited.}

%\distribution{DISTRIBUTION STATEMENT D: Distribution authorized to DoD components and U.S. DoD contractors only; Critical Technology; Dec 2009. Other requests for this document shall be referred to The Office of Naval Research (code 32). Destroy by any method that will prevent disclosure of contents or reconstruction of the document.}

% required
\title{Project Annual Report}

% required: always starts with N00014
\awardnumber{N00014-000-0000}

% optional
\projecturl{\url{https://github.com/samhu1989/DevBundle/}}

%% Meat of the document goes here
\begin{document}
    
% The author prefers apalike, but it doesn't work when the year is missing.
% This causes problems for the fake citation used to show that the
% references section is optional.
\bibliographystyle{abbrvnat}

% print the title/author frontmatter
\maketitle

\section{LONG-TERM GOALS}
Publish a Class-A~Paper.
\section{OBJECTIVES}
Our intension is to develop a fully automatic and unsupervised method to recover the object information (object shape and object-level segmentation) from a set of point clouds. We further assume that this set of point clouds contain the same set of objects but differ in the object motion.

\section{APPROACH}
This year we change the approach from an interleaving of region-grow and graph-cut to a EM based iteration framework. The advantage of this framework is that it comes with a more clear formulation (GMM based generative model) of the problem and it provide some guarantee of convergency. Under this framework we focuses on further improve our procedure to avoid local minimums.\\
Firstly, we try to start with a better initialization.\\
This is attempt is fanally dropped due to following reasons:\\
1. We don't know how "good" the initializations need to be.\\
2. It is not reasonable to ask for a good initializations from the users.A good enough initialization may need as much interaction (mannual annotation) as before.\\
3. Even with a pretty good initialization on correspondence, it works on some data but don't work on others.\\
Secondly, we try to add the features as constraints inside the iteration(lead to a bilateral GMM model).\\
This is dropped due to following reasons:\\
1. It is hard to find a consistent feature on noisy point clouds of indoor objects.\\
2. When given a pretty good feature (generated according to human annotation, not a reasonable way to extract feature but can help with the experiment), the result is better but still far from the ground truth object. It will form some odd objects that split objects and merge to other parts from different objects.\\
Now, we try to add assumptions to the shape of the objects. we plan to enforce such assumptions using primitive shapes.
I am planning to start with rectangle plates as primitives. I anticipate a group of five plates should be a good reprensentation for indoor objects.\\
This change will lead to a new method for the caculation of prior probability and update of objects.
\section{WORK COMPLETED}
1. I have done a series of experiments to exploration and verify different approahes.\\
a.JRCS baseline(Figure~\ref{fig:gmm})\\
b.JRCS bilateral(Figure~\ref{fig:biagmm})\\
b.JRCS primitive(Figure~\ref{fig:plate})\\
2. Along with it I have implemented a series of algorithms for feature extraction, segmentation, optimization, registration.\\
a. Normalized cut(Figure~\ref{fig:ncut})\\
\begin{figure}
	\begin{center}
		\includegraphics[width=\linewidth]{images/ncut.png}
	\end{center}
	\caption{Normalized Cut}
	\label{fig:ncut}
\end{figure}
b. Spectral analysis tool(Figure~\ref{fig:spec})\\
I have used this tool to explore the possibility of a spectral feature that is consistent among frames. However, the effort didn't pay off.
\begin{figure}
	\begin{center}
		\includegraphics[width=\linewidth]{images/spec.png}
	\end{center}
	\caption{Spectral Analysis Tool:\\This tool was used to explore the spectral feature that is consistent among frames, however the feature I found is not significant when the point clouds comes with noise and the correspondence is not accurate}
	\label{fig:spec}
\end{figure}
c. Feature generation according to annotation(Figure~\ref{fig:fgen})\\
\begin{figure}
	\begin{center}
		\includegraphics[width=\linewidth]{images/fgen.png}
	\end{center}
	\caption{Feature Generation According to Annotation}
	\label{fig:fgen}
\end{figure}
d. Mannually annotation tool(Figure~\ref{fig:ann})\\
This was actually done for Lin Yanting, but he eventually didn't use it. Since mannually annotate ground truth takes too much time.
\begin{figure}
	\begin{center}
		\includegraphics[width=\linewidth]{images/ann.png}
	\end{center}
	\caption{Mannually Annotation Tool}
	\label{fig:ann}
\end{figure}
3. I have implemented a series of tools for visualization, data generation, annotation and debugging.\\
\section{RESULTS}
1. Baseline (GMM)(Figure~\ref{fig:gmm})\\
\begin{figure}
	\begin{center}
		\includegraphics[width=\linewidth]{images/gmm.png}
	\end{center}
	\caption{Baseline Result}
	\label{fig:gmm}
\end{figure}
2. Bilateral GMM + Feature By Annotation(Figure~\ref{fig:biagmm})\\
The feature used is generated with the tool shown in Figure~\ref{fig:fgen}. It is basically gedesic distance to several landmark points(the number of points is the dimession of this feature). These points are mannually picked and corresponding to some semantic feature positions\\
\begin{figure}
	\begin{center}
		\includegraphics[width=\linewidth]{images/biagmm.png}
	\end{center}
	\caption{Bilateral Result}
	\label{fig:biagmm}
\end{figure}
3. Tests On the Plate as Primitive Shape
\begin{figure}
	\includegraphics[width=0.425\linewidth]{images/plate_a.png}
	\includegraphics[width=0.575\linewidth]{images/plate_b.png}
	\caption{Tests on the Plate as Primitive Shape: this is something under implementation}
	\label{fig:plate}
\end{figure}
\section{IMPACT/APPLICATIONS}

\section{TRANSITIONS}

%% If none, so state
\section{RELATED PROJECTS}

\section{PUBLICATIONS}

\end{document}